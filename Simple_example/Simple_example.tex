\documentclass{article}\usepackage[]{graphicx}\usepackage[]{color}
%% maxwidth is the original width if it is less than linewidth
%% otherwise use linewidth (to make sure the graphics do not exceed the margin)
\makeatletter
\def\maxwidth{ %
  \ifdim\Gin@nat@width>\linewidth
    \linewidth
  \else
    \Gin@nat@width
  \fi
}
\makeatother

\definecolor{fgcolor}{rgb}{0.345, 0.345, 0.345}
\newcommand{\hlnum}[1]{\textcolor[rgb]{0.686,0.059,0.569}{#1}}%
\newcommand{\hlstr}[1]{\textcolor[rgb]{0.192,0.494,0.8}{#1}}%
\newcommand{\hlcom}[1]{\textcolor[rgb]{0.678,0.584,0.686}{\textit{#1}}}%
\newcommand{\hlopt}[1]{\textcolor[rgb]{0,0,0}{#1}}%
\newcommand{\hlstd}[1]{\textcolor[rgb]{0.345,0.345,0.345}{#1}}%
\newcommand{\hlkwa}[1]{\textcolor[rgb]{0.161,0.373,0.58}{\textbf{#1}}}%
\newcommand{\hlkwb}[1]{\textcolor[rgb]{0.69,0.353,0.396}{#1}}%
\newcommand{\hlkwc}[1]{\textcolor[rgb]{0.333,0.667,0.333}{#1}}%
\newcommand{\hlkwd}[1]{\textcolor[rgb]{0.737,0.353,0.396}{\textbf{#1}}}%

\usepackage{framed}
\makeatletter
\newenvironment{kframe}{%
 \def\at@end@of@kframe{}%
 \ifinner\ifhmode%
  \def\at@end@of@kframe{\end{minipage}}%
  \begin{minipage}{\columnwidth}%
 \fi\fi%
 \def\FrameCommand##1{\hskip\@totalleftmargin \hskip-\fboxsep
 \colorbox{shadecolor}{##1}\hskip-\fboxsep
     % There is no \\@totalrightmargin, so:
     \hskip-\linewidth \hskip-\@totalleftmargin \hskip\columnwidth}%
 \MakeFramed {\advance\hsize-\width
   \@totalleftmargin\z@ \linewidth\hsize
   \@setminipage}}%
 {\par\unskip\endMakeFramed%
 \at@end@of@kframe}
\makeatother

\definecolor{shadecolor}{rgb}{.97, .97, .97}
\definecolor{messagecolor}{rgb}{0, 0, 0}
\definecolor{warningcolor}{rgb}{1, 0, 1}
\definecolor{errorcolor}{rgb}{1, 0, 0}
\newenvironment{knitrout}{}{} % an empty environment to be redefined in TeX

\usepackage{alltt}
\usepackage[sc]{mathpazo}
\usepackage[T1]{fontenc}
\usepackage{geometry}
\geometry{verbose,tmargin=2.5cm,bmargin=2.5cm,lmargin=2.5cm,rmargin=2.5cm}
\setcounter{secnumdepth}{2}
\setcounter{tocdepth}{2}
\usepackage{url}
\usepackage[unicode=true,pdfusetitle,
 bookmarks=true,bookmarksnumbered=true,bookmarksopen=true,bookmarksopenlevel=2,
 breaklinks=false,pdfborder={0 0 1},backref=false,colorlinks=false]
 {hyperref}
\hypersetup{
 pdfstartview={XYZ null null 1}}


\newcommand{\R}{\texttt{R} }
\IfFileExists{upquote.sty}{\usepackage{upquote}}{}
\begin{document}
\title{Testing a simple example for the inapplicability problem}

\author{Thomas Guillerme \\ t.guillerme@imperial.ac.uk}


\maketitle

This vignette aims to test which algorithms will pick up the signal of a really simple matrix with a clear signal.
We are going to test a simple 12 taxa matrix with a minimum parsimony of 18 steps and a ``stairs'' topology (i.e. (t1, (t2, (t3, etc.))); ).

We are going to test with \texttt{tnt} \texttt{inapplicable R} package and \texttt{phangorn R} package.

\section{Before starting}

We are also going to need the \texttt{IterativeAlgo} ``package'' that contains some utilities functions.

\begin{knitrout}
\definecolor{shadecolor}{rgb}{0.969, 0.969, 0.969}\color{fgcolor}\begin{kframe}
\begin{alltt}
\hlkwa{if}\hlstd{(}\hlopt{!}\hlkwd{require}\hlstd{(devtools))} \hlkwd{install.packages}\hlstd{(}\hlstr{"devtools"}\hlstd{)}
\hlkwd{install_github}\hlstd{(}\hlstr{"TGuillerme/Parsimony_Inapplicable/IterativeAlgo"}\hlstd{)}
\end{alltt}
\end{kframe}
\end{knitrout}

These are the required packages in \texttt{R}.

\begin{knitrout}
\definecolor{shadecolor}{rgb}{0.969, 0.969, 0.969}\color{fgcolor}\begin{kframe}
\begin{alltt}
\hlkwd{library}\hlstd{(IterativeAlgo)}
\end{alltt}


{\ttfamily\noindent\color{warningcolor}{\#\# Warning in .doLoadActions(where, attach): trying to execute load actions without 'methods' package}}\begin{alltt}
\hlkwa{if}\hlstd{(}\hlopt{!}\hlkwd{require}\hlstd{(ape))} \hlkwd{install.packages}\hlstd{(}\hlstr{"ape"}\hlstd{)}
\hlkwa{if}\hlstd{(}\hlopt{!}\hlkwd{require}\hlstd{(phangorn))} \hlkwd{install.packages}\hlstd{(}\hlstr{"phangorn"}\hlstd{)}
\end{alltt}


{\ttfamily\noindent\itshape\color{messagecolor}{\#\# Loading required package: phangorn}}\begin{alltt}
\hlkwa{if}\hlstd{(}\hlopt{!}\hlkwd{require}\hlstd{(phyclust))} \hlkwd{install.packages}\hlstd{(}\hlstr{"phyclust"}\hlstd{)}
\end{alltt}


{\ttfamily\noindent\itshape\color{messagecolor}{\#\# Loading required package: phyclust}}\begin{alltt}
\hlkwa{if}\hlstd{(}\hlopt{!}\hlkwd{require}\hlstd{(diversitree))} \hlkwd{install.packages}\hlstd{(}\hlstr{"diversitree"}\hlstd{)}
\end{alltt}


{\ttfamily\noindent\itshape\color{messagecolor}{\#\# Loading required package: diversitree}}\begin{alltt}
\hlkwa{if}\hlstd{(}\hlopt{!}\hlkwd{require}\hlstd{(inapplicable))} \hlkwd{install.packages}\hlstd{(}\hlstr{"inapplicable"}\hlstd{)}
\end{alltt}


{\ttfamily\noindent\itshape\color{messagecolor}{\#\# Loading required package: inapplicable}}

{\ttfamily\noindent\itshape\color{messagecolor}{\#\# Loading required package: parallel}}\end{kframe}
\end{knitrout}

\section{Loading the data}

The data is a simple 18 step matrix that should result in a ladderised tree.
The matrix is available at this link \url{https://github.com/TGuillerme/Parsimony_Inapplicable/blob/master/Simple_example/18step_matrix.csv}.

\begin{knitrout}
\definecolor{shadecolor}{rgb}{0.969, 0.969, 0.969}\color{fgcolor}\begin{kframe}
\begin{alltt}
\hlcom{## Reading in the matrix}
\hlstd{matrix} \hlkwb{<-} \hlkwd{as.matrix}\hlstd{(}\hlkwd{read.csv}\hlstd{(}\hlstr{"18step_matrix.csv"}\hlstd{,} \hlkwc{header} \hlstd{= F,} \hlkwc{row.names}\hlstd{=}\hlnum{1}\hlstd{))}

\hlcom{## Reading in the expected tree}
\hlstd{expected_tree} \hlkwb{<-} \hlkwd{read.tree}\hlstd{(}
    \hlkwc{text} \hlstd{=} \hlstr{"(t1,(t2,(t3,(t4,(t5,(t6,(t7,(t8,(t9,(t10,(t11,t12)))))))))));"}\hlstd{)}
\end{alltt}
\end{kframe}
\end{knitrout}


\section{Setting up the data to be used by \texttt{inapplicable}}

A couple of transformations have to be made in order to feed the data to the \texttt{inapplicable} package:

\begin{knitrout}
\definecolor{shadecolor}{rgb}{0.969, 0.969, 0.969}\color{fgcolor}\begin{kframe}
\begin{alltt}
\hlcom{## Creating the contrast matrix (function from IterativeAlgo)}
\hlstd{contrast_matrix} \hlkwb{<-} \hlkwd{get.contrast.matrix}\hlstd{(matrix)}

\hlcom{## Create the phyDat object}
\hlstd{matrix_phyDat} \hlkwb{<-} \hlkwd{phyDat}\hlstd{(matrix,} \hlkwc{type} \hlstd{=} \hlstr{'USER'}\hlstd{,} \hlkwc{contrast} \hlstd{= contrast_matrix)}

\hlcom{## Optimise the data for the inapplicable package analysis}
\hlstd{matrix_inap} \hlkwb{<-} \hlkwd{prepare.data}\hlstd{(matrix_phyDat)}

\hlcom{## Specify the names of the outgroup taxa}
\hlstd{outgroup} \hlkwb{<-} \hlkwd{c}\hlstd{(}\hlstr{'t1'}\hlstd{)}

\hlcom{## Set random seed for generating the tree}
\hlkwd{set.seed}\hlstd{(}\hlnum{1}\hlstd{)}

\hlcom{## Generate a random tree}
\hlstd{rand_tree} \hlkwb{<-} \hlkwd{rtree}\hlstd{(}\hlkwd{nrow}\hlstd{(matrix),} \hlkwc{tip.label} \hlstd{=} \hlkwd{rownames}\hlstd{(matrix),} \hlkwc{br} \hlstd{=} \hlkwa{NULL}\hlstd{)}

\hlcom{## Generate a NJ tree (with 0 branch length)}
\hlstd{nj_tree} \hlkwb{<-} \hlkwd{root}\hlstd{(}\hlkwd{nj}\hlstd{(}\hlkwd{dist.hamming}\hlstd{(matrix_phyDat)), outgroup,} \hlkwc{resolve.root} \hlstd{=} \hlnum{TRUE}\hlstd{)}
\hlstd{nj_tree}\hlopt{$}\hlstd{edge.length} \hlkwb{<-} \hlkwa{NULL}

\hlcom{## Generate a list of trees}
\hlstd{trees} \hlkwb{<-} \hlkwd{list}\hlstd{(expected_tree, rand_tree, expected_tree)}
\hlkwd{class}\hlstd{(trees)} \hlkwb{<-} \hlstr{"multiPhylo"}
\end{alltt}
\end{kframe}
\end{knitrout}

\section{Calculating parsimony}

We can then calculate the parsimony score using the simple Fitch algorithm from \texttt{phytools} and the one from \texttt{inapplicable} using the expected tree, the random one and the NJ calculated one.

\begin{knitrout}
\definecolor{shadecolor}{rgb}{0.969, 0.969, 0.969}\color{fgcolor}\begin{kframe}
\begin{alltt}
\hlcom{## Calculate the normal fitch parsimony score using the 3 trees}
\hlstd{parsimony_fitch} \hlkwb{<-} \hlkwd{unlist}\hlstd{(}\hlkwd{lapply}\hlstd{(trees, parsimony, matrix_phyDat,}
    \hlkwc{method} \hlstd{=} \hlstr{"fitch"}\hlstd{))}

\hlcom{## Calculate the parsimony score using the inapplicable algorithm}
\hlstd{parsimony_inapp} \hlkwb{<-} \hlkwd{unlist}\hlstd{(}\hlkwd{lapply}\hlstd{(trees, parsimony.inapp, matrix_inap))}

\hlcom{## Displaying the results}
\hlkwd{matrix}\hlstd{(}\hlkwd{c}\hlstd{(parsimony_fitch, parsimony_inapp),} \hlnum{2}\hlstd{,} \hlnum{3}\hlstd{,} \hlkwc{byrow} \hlstd{=} \hlnum{TRUE}\hlstd{,}
    \hlkwc{dimnames} \hlstd{=} \hlkwd{list}\hlstd{(}\hlkwd{c}\hlstd{(}\hlstr{"Fitch parsimony"}\hlstd{,} \hlstr{"Inapp parsimony"}\hlstd{),}
    \hlkwd{c}\hlstd{(}\hlstr{"Expecte_tree"}\hlstd{,} \hlstr{"Random_tree"}\hlstd{,} \hlstr{"NJ_tree"}\hlstd{)))}
\end{alltt}
\begin{verbatim}
##                 Expecte_tree Random_tree NJ_tree
## Fitch parsimony           30          79      30
## Inapp parsimony           18          31      18
\end{verbatim}
\end{kframe}
\end{knitrout}

We can also use the tree search algorithms from the \texttt{inapplicable} package.
These functions do not seem to work properly (at least on the version of the \texttt{inapplicable} package I have, they crash \texttt{R}).

\begin{knitrout}
\definecolor{shadecolor}{rgb}{0.969, 0.969, 0.969}\color{fgcolor}\begin{kframe}
\begin{alltt}
\hlcom{## Search for a better tree}
\hlstd{better_tree} \hlkwb{<-} \hlkwd{tree.search}\hlstd{(expected_tree, matrix_inap, outgroup)}

\hlcom{## Try a sectorial search (Goloboff, 1999)}
\hlstd{better_tree} \hlkwb{<-} \hlkwd{sectorial.search}\hlstd{(rand_tree, matrix_inap)}

\hlcom{## Try the parsimony ratchet (Nixon, 1999)}
\hlstd{better_tree} \hlkwb{<-} \hlkwd{pratchet.inapp}\hlstd{(expected_tree, matrix_inap, outgroup)}
\end{alltt}
\end{kframe}
\end{knitrout}

\end{document}
